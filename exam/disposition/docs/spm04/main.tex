\section{Asymmetrisk nøglekryptering og digitale certifikater}

\subsection{Læringsmål}

\begin{itemize}
	\item Redegør for og vis eksempler på asymmetrisk nøglekryptering, samt anvendelsen af
	asymmetrisk nøglekryptering på Internet.
	\item Samt forklar hvordan digitale certifikater udstedes og verificeres.
\end{itemize}

\subsection{Redegør for og vis eksempler på asymmetrisk nøglekryptering, samt anvendelsen af asymmetrisk nøglekryptering på Internet}\label{sec:asymmetric}
Figur~\ref{fig:asymmetrickeyenc} viser eksempel på asymmetrisk kryptering. Med asymmetrisk kryptering kan kun personen med den \textit{private nøgle} decryptere beskeden. 

Svare til at sende en boks med åben hængelås ud. Alle kan lukke boksen og sende den til dig, men kun du har nøglen til at låse op.

Er eksponentielt langsommere end symmetrisk kryptering.

\begin{figure}[H]
	\centering
	\includegraphics[width=0.7\linewidth]{figs/spm4/asymmetric_key_enc}
	\caption{Asymmetrisk kryptering}
	\label{fig:asymmetrickeyenc}
\end{figure}

\subsubsection{Matematikken}
Ligning~\ref{eq:asym} viser hvordan asymmetrisk kryptering virker. I ligningen er følgende forkortelser brugt: C: Ciphertext, P: Plaintext, E: Encryption rule, D: Decryption rule.

\begin{align}\label{eq:asym}
	C &= E(k_{priv}, P)\\
	P &= D(k_{priv}, E(k_{pub}, P))\\
	P &= D(k_{pub}, E(k_{priv}, P))
\end{align}

\subsubsection{Eksempel på RSA kryptering}
I Ligning~\ref{eq:rsa} er \textit{n} produktet af to primtal. Tallene \textit{d} og \textit{e} er primtallene.

\begin{align}\label{eq:rsa}
	Alice &= (n = 3233, e = 17)\\
	Bob &= (n = 3233, d = 413)\\
	Message &= 65\\
	Encrypted &= 65^{17}~mod~3233 = 2790\\
	Decrypted &= 2790^{413}~mod~3233 = 65
\end{align}

\subsubsection{Cryptographiv Primitives}
Disse grundlæggende blokke bruges til at lave krypteringsalgoritmer.

\paragraph{Substitution}
Udskifter tegn med andre. F.eks. $ABC \Rightarrow 123$

\paragraph{Transposition}
Flytter tegn i strengen. F.eks. $ABC \Rightarrow BCA$

\paragraph{Confusion}
Ændringen af ét tegn ændre hele den resulterende streng.

\paragraph{Diffusion}
Distribuere informationen fra ét tegn ud i hele cifferteksten.

\subsubsection{Kryptering på internettet}
TLS er opgraderingen af SSL. Disse anvendes i \textit{transport} laget af netværkskommunikation. 

I applikationslaget hedder det \textit{HTTPS} når forbindelsen er beskyttet af TLS.

\subsection{Samt forklar hvordan digitale certifikater udstedes og verificeres}
Certifikater indeholder: en public key, en identitet og er signeret af en \textit{certificate authority}.

\subsubsection{Udstedelse}
Når et certifikat udsteder bruges flowet vist i Figur~\ref{fig:cert-issue}.

\begin{figure}[H]
	\centering
	\includegraphics[width=0.8\linewidth]{figs/spm4/cert-issue}
	\caption{Udstedelse af et certifikat.}
	\label{fig:cert-issue}
\end{figure}

\subsubsection{Validering}
Ved validering af certifikater bruges flowt vist i Figur~\ref{fig:cert-validate}.

\begin{figure}[H]
	\centering
	\includegraphics[width=0.9\linewidth]{figs/spm4/cert-validate}
	\caption{Validering af et certifikat.}
	\label{fig:cert-validate}
\end{figure}
