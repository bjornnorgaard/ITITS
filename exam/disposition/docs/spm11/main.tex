\section{It sikkerhed i relation til netværk}

\subsection{Læringsmål}

\begin{itemize}
	\item Redegør for typiske sårbarheder og trusler relateret til netværk.
	\item Giv i den forbindelse en uddybende gennemgang af sikkerhedsproblematikker ved	brug af WiFi.
	\item Forklar Denial-of-service attacks, og hvordan man kan beskytte sig mod disse.
	\item Redegør for de forskellige typer af firewalls.
	\item Forklar hvad ”Intrusion detection and prevention systems” er og hvordan de virker.
\end{itemize}

\subsection{Redegør for typiske sårbarheder og trusler relateret til netværk}
Netværk kan være usikre på mange måder. Åbne porte og ikke-krypteret trådlås trafik er nogle.

\subsection{Giv i den forbindelse en uddybende gennemgang af sikkerhedsproblematikker ved brug af WiFi}
Hvis ikke trafikken er krypteret vil alt kunne læses af enhver der lytter. 

\begin{itemize}
	\item \textbf{Identity theft (MAC spoofing)} Angriber lytter på netværket og finder en computers MAC adresse og angriber ved at udgive sig for at være denne.
	\item \textbf{Man-in-the-middel attacks} Angriber lavet sit eget AP og får en bruger til at logge på.
	\item \textbf{Denial of service (DoS)} Angriber sender en masse traffik til målet.
\end{itemize}

\subsection{Forklar Denial-of-service attacks, og hvordan man kan beskytte sig mod disse}
Et DoS angreb skal gøre en tjeneste/service utilgængelig eller dårligere for legitime brugere. Dette kan ske ved at manipulere forskellige sårbarheden, ikke kun netværk.

\paragraph{Utilstrækkelige ressourcer} gør DoS angreb nemmere at udgøre og kan også resultere i DoS lignende symptomer, uden at et faktisk angreb er udført\footnote{Black Friday}.

\paragraph{Public relations} kan motivere DOS angreb.

\subsubsection{Typer af DoS angreb}

\paragraph{DoS} bruger en enkelt forbindelse.

\paragraph{DDoS} anvender flere computere eller botnets til at udføre samtidige DOS angreb.

\paragraph{Amplified DDos} udnytter at ikke alle tjeneste validere indkommende request. Herved kan en angriber sende et request til en service, hvor svaradressen er målets\footnote{NTP}.

\paragraph{Slow Loris} åbner mange samtidige forbindelser med serveren og vedligeholder disse med et minimum af aktivitet for at holde forbindelsen oppe.

\subsubsection{Mitigation}
DoS er svære at beskytte sig imod, nogle ting kan man dog gøre:

\begin{itemize}
	\item Højere og vildere servere.
	\item Brug en eller flere reverse proxies for at beskytte servere og fjerne single-point-of-failure. Det kan eksempelvis være en service som CloudFlare.
	\item Begrænse antallet af forbindelser som en IP adresse må oprette. Frasortere særlig langsom trafik eller sænke timeout tiden for requests.
\end{itemize}

\subsection{Redegør for de forskellige typer af firewalls}
\begin{itemize}
	\item \textbf{Standalone}
	\begin{itemize}
		\item ModSecurity
		\item Wallarm
		\item UrlScan
	\end{itemize}
	\item \textbf{Cloud services}
	\begin{itemize}
		\item CloudFlare
		\item Sucuri
	\end{itemize}
	\item \textbf{Network applications}
	\begin{itemize}
		\item Barracuda Networks
		\item Citrix NetScaler
	\end{itemize}
\end{itemize}

\subsection{Forklar hvad ”Intrusion detection and prevention systems” er og hvordan de virker}
IDS overvåger netværk eller host på udvalgte steder. Erstatter ikke korrekt brug af autentifikation.

\subsubsection{Detection}
Overvåger og sender en advarsel hvis trusler detekteres.

\paragraph{Netværk detection}
På et netværk analysere den al trafik og matcher mod kendte sårbarheder. Herefter kan en advarsel sendes til en administrator. Dette kan have dårlige konsekvenser for performance.

\paragraph{Host detection}
Holder øje med maskiner, som regel mission kritiske da deres konfiguration ikke forventes at ændres.

\subsubsection{Prevention}
Står in-line i systemet. 

Disse systemer reagere på flere måder: stopper angrebet, ændre miljøet (rekonfiguration af firewall) eller ved at ændre indholdet i pakken.
